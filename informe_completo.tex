\documentclass[10pt,twocolumn]{article}
\usepackage[utf8]{inputenc}
\usepackage[spanish]{babel}
\usepackage{amsmath}
\usepackage{amsfonts}
\usepackage{amssymb}
\usepackage{graphicx}
\usepackage{geometry}
\usepackage{cite}
\usepackage{float}
\usepackage{hyperref}
\usepackage{xcolor}
\usepackage{abstract}
\usepackage{multicol}

\geometry{a4paper, margin=1.8cm, columnsep=0.6cm}

\begin{document}

\title{\textbf{Modelado de Flujo Incompresible Utilizando las Ecuaciones de Navier-Stokes}}

\author{
Juan Esteban Ortiz, Brayan Camilo Urrea Jurado, Ervin Caravali Ibarra\\
\small Facultad de Ingeniería, Escuela de Ingeniería de Sistemas y Computación\\
\small Universidad del Valle, Santiago de Cali, Colombia\\
\small \textit{Dirigido a: Maria Patricia Trujillo Uribe}
}

\date{Noviembre 2025}

\maketitle

\begin{abstract}
\noindent Este trabajo presenta una implementación computacional completa para la simulación de flujo incompresible bidimensional en un canal con obstáculos, utilizando las ecuaciones de Navier-Stokes. Se empleó el método de diferencias finitas centradas para discretizar el dominio en una malla de $50 \times 5$ unidades, generando un sistema algebraico no lineal que se resolvió mediante el método de Newton-Raphson. Para el sistema lineal interno se implementó el Gradiente Conjugado aplicado a la matriz simetrizada $\mathbf{J}^T\mathbf{J}$, aprovechando la estructura dispersa del Jacobiano. Se desarrollaron dos versiones del código: V4 con interpolación spline cúbica natural 1D manual, y V4.1 con interpolación spline bicúbica de SciPy, comparando ambos enfoques. Los resultados muestran convergencia en 97 iteraciones con número de Reynolds $Re = 5.00$ (régimen laminar) y número de condición $\kappa \approx 3.07$ (bien condicionada).
\end{abstract}

\noindent\textbf{Palabras clave:} Navier-Stokes, diferencias finitas, Newton-Raphson, gradiente conjugado, spline cúbico, spline bicúbico, flujo incompresible, CFD

\section{Introducción}

El modelado de flujos incompresibles mediante las ecuaciones de Navier-Stokes constituye un problema fundamental en dinámica de fluidos computacional (CFD). Estas ecuaciones, que describen el movimiento de fluidos viscosos, forman un sistema de ecuaciones diferenciales parciales no lineales cuya solución analítica es prácticamente imposible para geometrías complejas \cite{landau2018}.

Este trabajo aborda la simulación del flujo incompresible bidimensional alrededor de dos vigas rígidas ubicadas dentro de un canal rectangular. El objetivo principal es determinar el campo de velocidades en el dominio, considerando la interacción del flujo con los obstáculos internos y la aplicación de condiciones de frontera específicas.

La metodología empleada combina: (i) discretización espacial por diferencias finitas centradas, (ii) resolución del sistema no lineal mediante Newton-Raphson con malla optimizada de $5 \times 50$ celdas, (iii) solución del sistema lineal interno con Gradiente Conjugado aplicado al sistema simetrizado, y (iv) dos enfoques de interpolación para visualización mejorada: spline cúbico natural 1D manual (V4) y spline bicúbico de SciPy (V4.1).

\section{Descripción del Problema}

\subsection{Configuración Física del Dominio}

El dominio corresponde a un canal rectangular con dos regiones internas que representan vigas rígidas. Las condiciones de frontera aplicadas fueron:

\begin{itemize}
\item \textbf{Entrada} ($x=0$): $u = 1.0$ m/s, $v = 0$
\item \textbf{Frontera superior}: velocidad horizontal constante $u = 1.0$ m/s
\item \textbf{Frontera inferior}: no deslizamiento $u = v = 0$
\item \textbf{Salida} ($x=L$): gradiente nulo $\partial u/\partial x = 0$, $\partial v/\partial x = 0$
\item \textbf{Superficies de vigas}: no deslizamiento $u = v = 0$
\end{itemize}

\begin{figure}[H]
    \centering
    \includegraphics[width=0.95\linewidth]{figuras/dominio_real.png}
    \caption{Dominio computacional y condiciones de frontera. El canal tiene dimensiones $50 \times 5$ con dos obstáculos rectangulares.}
    \label{fig:dominio}
\end{figure}

\subsection{Ecuaciones Gobernantes}

Las ecuaciones de Navier-Stokes para fluido incompresible en estado estacionario bidimensional se expresan como:

\textbf{Ecuación de continuidad:}
\begin{equation}
\frac{\partial v_x}{\partial x} + \frac{\partial v_y}{\partial y} = 0
\end{equation}

\textbf{Ecuación de momento (componente x):}
\begin{equation}
\nu\nabla^2 v_x = v_x\frac{\partial v_x}{\partial x} + v_y\frac{\partial v_x}{\partial y} + \frac{1}{\rho}\frac{\partial P}{\partial x}
\end{equation}

\textbf{Ecuación de momento (componente y):}
\begin{equation}
\nu\nabla^2 v_y = v_x\frac{\partial v_y}{\partial x} + v_y\frac{\partial v_y}{\partial y} + \frac{1}{\rho}\frac{\partial P}{\partial y}
\end{equation}

donde $v_x$ y $v_y$ son las componentes de velocidad, $\nu = 1$ m$^2$/s es la viscosidad cinemática, $\rho = 1$ kg/m$^3$ es la densidad, y $P$ es la presión. Se asumió estado estacionario y presión independiente de densidad y temperatura.

\subsection{Construcción de la Malla Computacional}

Con el fin de aproximar numéricamente el comportamiento del fluido, el dominio continuo se discretizó mediante una malla uniforme de diferencias finitas. Esta malla permite representar el modelo como una matriz de coordenadas donde se almacenan las componentes de velocidad del flujo $[i,j]$.

\textbf{Parámetros de la malla:}
\begin{itemize}
\item $N_x = 50$ nodos en dirección horizontal
\item $N_y = 5$ nodos en dirección vertical  
\item Tamaño de celda homogéneo $h = 1$
\item Total de nodos: $250$
\item Incógnitas (nodos internos): $134$
\end{itemize}

La presencia de obstáculos sólidos se incorpora anulando la velocidad en los nodos correspondientes, mientras que en los bordes se aplican las condiciones previamente definidas.

\section{Discretización de las Ecuaciones}

\subsection{Método de Diferencias Finitas Centradas}

El problema continuo se transformó en un sistema algebraico mediante la aplicación del Método de Diferencias Finitas Centradas \cite{kincaid2002}. Las derivadas parciales se aproximan como:

\begin{equation}
\frac{\partial v}{\partial x}\bigg|_{i,j} \approx \frac{v_{i+1,j} - v_{i-1,j}}{2h}
\end{equation}

\begin{equation}
\frac{\partial^2 v}{\partial x^2}\bigg|_{i,j} \approx \frac{v_{i+1,j} - 2v_{i,j} + v_{i-1,j}}{h^2}
\end{equation}

\subsection{Ecuación Discretizada Base}

Partiendo de la ecuación de Navier-Stokes para la componente horizontal de la velocidad, la aproximación mediante diferencias finitas centradas con $h=1$ da lugar a la siguiente expresión general (ver Anexo A para derivación completa):

\begin{multline}
v^x_{i,j} = \frac{1}{4}\left[v^x_{i+1,j} + v^x_{i-1,j} + v^x_{i,j+1} + v^x_{i,j-1}\right] \\
- \frac{1}{2}v^x_{i,j}\left[v^x_{i+1,j} - v^x_{i-1,j}\right] \\
- \frac{1}{2}v^y_{i,j}\left[v^x_{i,j+1} - v^x_{i,j-1}\right]
\label{eq:discretizada}
\end{multline}

Esta ecuación relaciona la velocidad en el nodo $(i,j)$ con la velocidad de sus cuatro vecinos inmediatos, incorporando los términos convectivos no lineales.

\subsection{Múltiples Patrones de Ecuaciones}

El análisis permitió identificar y codificar \textbf{nueve patrones de ecuaciones distintos}, cubriendo:

\begin{enumerate}
\item Nodos interiores (ecuación base completa)
\item Borde izquierdo (entrada)
\item Borde derecho (salida)
\item Borde superior
\item Borde inferior
\item Esquinas (4 casos)
\item Nodos adyacentes a vigas (superior e inferior)
\end{enumerate}

Cada patrón actúa como una regla de actualización específica durante la simulación, garantizando que las condiciones de frontera se cumplan correctamente.

\section{Método de Newton-Raphson}

\subsection{Malla Optimizada}

Para resolver el sistema no lineal generado, se aplicó el método de Newton-Raphson. Con el fin de reducir el costo computacional, el procedimiento se implementó sobre una \textbf{malla optimizada de $5 \times 50$ celdas}, con tamaño $h=8$, manteniendo la estructura física del dominio.

Esta optimización reduce significativamente el número de incógnitas de $250$ a $134$, acelerando cada iteración sin comprometer la captura de las características principales del flujo.

\subsection{Formulación del Sistema No Lineal}

Tras discretizar los términos, se obtuvo un sistema algebraico donde aparecen productos de velocidades, introduciendo términos no lineales:

\begin{equation}
\mathbf{F}(\mathbf{X}) = \mathbf{0}
\end{equation}

donde $\mathbf{X}$ es el vector de velocidades y $\mathbf{F}$ es el vector de funciones del sistema. El método de Newton-Raphson itera según:

\begin{equation}
\mathbf{X}^{(k+1)} = \mathbf{X}^{(k)} + \mathbf{H}^{(k)}
\end{equation}

donde el incremento $\mathbf{H}^{(k)}$ se obtiene resolviendo:

\begin{equation}
\mathbf{J}(\mathbf{X}^{(k)})\mathbf{H}^{(k)} = -\mathbf{F}(\mathbf{X}^{(k)})
\end{equation}

\subsection{La Matriz Jacobiana}

El Método de Newton-Raphson requiere la evaluación de la matriz Jacobiana, que se obtuvo derivando cada ecuación respecto a las variables de velocidad. Las derivadas parciales de la ecuación (\ref{eq:discretizada}) generan la estructura del Jacobiano (ver Anexo B para cálculo detallado).

\textbf{Propiedades clave:}
\begin{itemize}
\item \textbf{Altamente dispersa}: Exactamente 5 elementos no nulos por fila
\item \textbf{Estructura de banda}: Debido al ordenamiento secuencial
\item \textbf{No simétrica}: Por los términos convectivos
\item \textbf{No diagonal dominante}: Viola la condición en múltiples filas
\item \textbf{Definida positiva}: Valores singulares positivos
\end{itemize}

\begin{figure}[H]
\centering
\includegraphics[width=0.85\columnwidth]{figuras/estructura_jacobiano.png}
\caption{Estructura dispersa de la matriz Jacobiana ($134 \times 134$) mostrando el patrón de cinco diagonales característico.}
\label{fig:jacobiano}
\end{figure}

\subsection{Vector Inicial de Velocidades}

Se definió un vector inicial de velocidades decreciente de izquierda a derecha, coherente con el comportamiento esperado del flujo:

\begin{equation}
v^x_{i,j}(0) = 1.0 - \frac{i}{N_x}
\end{equation}

Este perfil inicial favorece la convergencia del método al estar cerca de la solución física esperada.

\section{Solución del Sistema Lineal}

\subsection{Análisis de Propiedades}

Dado que la matriz $\mathbf{J}$ no es simétrica ni garantiza diagonal dominante, la estrategia más robusta es utilizar el \textbf{Método de Gradiente Conjugado (CG)} \cite{kincaid2002} aplicado al sistema simetrizado:

\begin{equation}
\mathbf{J}^T\mathbf{J}\mathbf{H} = -\mathbf{J}^T\mathbf{F}
\end{equation}

La nueva matriz $\mathbf{J}^T\mathbf{J}$ es intrínsecamente simétrica y definida positiva, compatible con CG.

\subsection{Número de Condición}

Para evaluar la estabilidad numérica, se calculó el número de condición usando la norma 2:

\begin{equation}
\kappa(\mathbf{J}) = \|\mathbf{J}\|_2 \|\mathbf{J}^{-1}\|_2 \approx 3.07
\end{equation}

Este valor indica que la matriz está \textbf{bien condicionada}, no amplificando significativamente los errores de redondeo, lo que favorece la estabilidad del Gradiente Conjugado.

\subsection{Comparación de Métodos Iterativos}

Se implementaron y compararon cinco métodos iterativos para resolver el sistema lineal:

\begin{table}[H]
\centering
\small
\begin{tabular}{|l|c|c|}
\hline
\textbf{Método} & \textbf{Convergió} & \textbf{Iteraciones} \\
\hline
Jacobi & No & 1 \\
Gauss-Seidel & No & 1 \\
Richardson & Sí & 14 \\
Gradient Descent & Sí & 18 \\
Conjugate Gradient & Sí & 97 \\
\hline
\end{tabular}
\caption{Comparación de métodos iterativos. Richardson fue el más rápido, pero CG es más robusto para sistemas generales.}
\end{table}

\section{Resultados y Análisis}

\subsection{Convergencia del Método}

El método de Newton-Raphson convergió exitosamente en \textbf{97 iteraciones} con tolerancia $\|\mathbf{F}\| < 10^{-6}$, mostrando comportamiento cuadrático característico. El tiempo de ejecución fue aproximadamente 0.2 segundos en la malla optimizada.

\subsection{Campo de Velocidades - Versión V4}

La versión V4 implementa interpolación mediante \textbf{splines cúbicos naturales 1D manual} en dos etapas (horizontal y vertical), aumentando la resolución de $5 \times 50$ a $50 \times 500$ puntos.

\begin{figure}[H]
\centering
\includegraphics[width=\columnwidth]{figuras/malla_computacional.png}
\caption{Campo de velocidades V4 sin interpolación en la malla original $5 \times 50$, mostrando la discretización y los obstáculos.}
\label{fig:v4_sin}
\end{figure}

\begin{figure}[H]
\centering
\includegraphics[width=\columnwidth]{figuras/campo_velocidades.png}
\caption{Campo de velocidades V4 con interpolación spline cúbica natural 1D manual. Se observa suavizado significativo y captura de gradientes.}
\label{fig:v4_con}
\end{figure}

\subsection{Campo de Velocidades - Versión V4.1}

La versión V4.1 utiliza \textbf{interpolación spline bicúbica de SciPy} (función \texttt{RectBivariateSpline}), que realiza interpolación bidimensional simultánea en lugar de dos etapas separadas.

\begin{figure}[H]
\centering
\includegraphics[width=\columnwidth]{figuras/campo_velocidades_v4_1_sin_spline.png}
\caption{Campo de velocidades V4.1 sin interpolación, idéntico a V4 en la malla base.}
\label{fig:v41_sin}
\end{figure}

\begin{figure}[H]
\centering
\includegraphics[width=\columnwidth]{figuras/campo_velocidades_v4_1_spline.png}
\caption{Campo de velocidades V4.1 con interpolación spline bicúbica de SciPy. Presenta mayor suavidad y continuidad que V4.}
\label{fig:v41_con}
\end{figure}

\subsection{Comparación de Métodos de Interpolación}

\textbf{Spline Cúbico Natural 1D Manual (V4):}
\begin{itemize}
\item Implementación en dos etapas (horizontal, luego vertical)
\item Control total sobre el algoritmo
\item Menor costo computacional
\item Puede presentar artefactos en las esquinas
\end{itemize}

\textbf{Spline Bicúbico SciPy (V4.1):}
\begin{itemize}
\item Interpolación bidimensional simultánea
\item Mayor suavidad y continuidad $C^2$
\item Optimizado y validado
\item Mejor captura de gradientes diagonales
\end{itemize}

Ambos métodos producen resultados físicamente coherentes, pero V4.1 ofrece mayor calidad visual y continuidad matemática.

\subsection{Análisis Físico del Flujo}

Los resultados muestran características físicamente coherentes:

\begin{itemize}
\item \textbf{Aceleración entre obstáculos}: Principio de Bernoulli
\item \textbf{Zonas de recirculación}: Aguas abajo de las vigas
\item \textbf{Gradientes suaves}: En regiones alejadas de obstáculos
\item \textbf{Conservación de masa}: Cumplida en toda la malla
\end{itemize}

\subsection{Líneas de Corriente}

Las líneas de corriente (streamlines) proporcionan una visualización intuitiva del patrón de flujo:

\begin{figure}[H]
\centering
\includegraphics[width=\columnwidth]{figuras/streamlines.png}
\caption{Líneas de corriente mostrando el patrón de flujo alrededor de los obstáculos. Se observan claramente las zonas de recirculación detrás de las vigas y la aceleración en regiones estrechas.}
\label{fig:streamlines}
\end{figure}

\subsection{Número de Reynolds}

El número de Reynolds calculado es:

\begin{equation}
Re = \frac{vL}{\nu} = \frac{1.0 \times 5}{1} = 5.00
\end{equation}

Este valor ($Re < 2300$) confirma \textbf{régimen laminar}, validando:
\begin{itemize}
\item Ausencia de turbulencia
\item Flujo ordenado y predecible
\item Capas que se deslizan suavemente
\item Validez del modelo sin términos turbulentos
\end{itemize}

\section{Descripción de Archivos del Proyecto}

Para facilitar la reproducibilidad y comprensión del código desarrollado, a continuación se detalla la función de cada script del proyecto y sus componentes principales:

\subsection{Scripts de Simulación Principal}

\textbf{\texttt{campo\_velocidadesV4.py}}: Implementa la solución numérica completa utilizando interpolación manual.
\begin{itemize}
    \item \texttt{interpolate\_cubic\_natural\_manual(V\_low\_res)}: Realiza la interpolación spline cúbica en dos etapas (1D horizontal y luego 1D vertical) para suavizar la malla de $5 \times 50$ a $50 \times 500$.
    \item \texttt{plot\_solution(...)}: Genera las visualizaciones finales del campo de velocidades.
    \item \texttt{analizar\_y\_mostrar\_resultados(...)}: Coordina la ejecución de los solvers iterativos y compara su rendimiento.
\end{itemize}

\textbf{\texttt{campo\_velocidadesV4.1.py}}: Versión optimizada que utiliza bibliotecas científicas para la interpolación.
\begin{itemize}
    \item \texttt{interpolate\_bicubic\_natural(V\_low\_res)}: Utiliza \texttt{scipy.interpolate.RectBivariateSpline} para una interpolación bidimensional más robusta y eficiente ($C^2$ continuo).
    \item \texttt{plot\_solution\_sin\_spline(...)}: Visualiza los resultados crudos directamente sobre la malla computacional de bajo orden.
\end{itemize}

\subsection{Scripts de Análisis y Visualización}

\textbf{\texttt{visualizar\_streamlines.py}}: Genera las líneas de corriente para visualizar la trayectoria del fluido.
\begin{itemize}
    \item \texttt{calcular\_campo\_vy(Vx\_matrix)}: Estima la componente vertical de la velocidad $v_y$ basándose en la ecuación de continuidad $\nabla \cdot \mathbf{v} = 0$.
    \item \texttt{crear\_streamlines\_interpolado(...)}: Genera el gráfico de líneas de corriente suave utilizando los campos de velocidad interpolados.
\end{itemize}

\textbf{\texttt{analisis\_reynolds.py}}: Calcula y valida los parámetros físicos de la simulación.
\begin{itemize}
    \item \texttt{calcular\_reynolds()}: Determina el número de Reynolds basado en la velocidad característica, longitud característica y viscosidad cinemática, confirmando el régimen laminar ($Re=5.00$).
\end{itemize}

\textbf{\texttt{independencia\_malla.py}}: Realiza el estudio de convergencia numérica.
\begin{itemize}
    \item \texttt{analizar\_independencia\_malla()}: Ejecuta la simulación con diferentes densidades de malla para verificar que la solución no depende del tamaño de discretización.
    \item \texttt{generar\_grafico\_convergencia(...)}: Grafica el error relativo vs. el número de nodos.
\end{itemize}

\subsection{Scripts de Generación de Figuras}

\textbf{\texttt{generar\_dominio.py}}: Crea el diagrama esquemático del dominio computacional (Fig. \ref{fig:dominio}), dibujando los obstáculos y etiquetas de condiciones de frontera mediante \texttt{matplotlib.patches}.

\textbf{\texttt{generar\_jacobiano.py}}: Visualiza la estructura de la matriz del sistema lineal (Fig. \ref{fig:jacobiano}), permitiendo inspeccionar el patrón de dispersión (sparsity pattern) de las cinco diagonales no nulas.

\subsection{Entorno de Ejecución}

Para garantizar la reproducibilidad de los resultados, todo el proyecto ha sido configurado para ejecutarse dentro de un \textbf{entorno virtual de Python (venv)}. Esto asegura el aislamiento de las dependencias y la compatibilidad de versiones.

\textbf{Requisitos:}
\begin{itemize}
    \item Python 3.8+
    \item Entorno virtual activado: \texttt{source venv/bin/activate}
    \item Dependencias principales: \texttt{numpy}, \texttt{scipy}, \texttt{matplotlib}
\end{itemize}

\textbf{Repositorio:} El código fuente completo está disponible en:\\
\url{https://github.com/ErvinCaraval/proyecto-final-simulacion.git}

\section{Conclusiones}

Este trabajo desarrolló exitosamente una implementación computacional completa para simular flujo incompresible bidimensional con las ecuaciones de Navier-Stokes. Los principales logros son:

\begin{enumerate}
\item La discretización por diferencias finitas centradas transformó efectivamente el problema continuo en un sistema algebraico con nueve patrones de ecuaciones.

\item El método de Newton-Raphson con malla optimizada ($5 \times 50$) demostró alta eficiencia, convergiendo en 97 iteraciones.

\item El Gradiente Conjugado aplicado al sistema simetrizado $\mathbf{J}^T\mathbf{J}$ fue robusto, aprovechando la estructura dispersa (4.76\% densidad).

\item El número de condición $\kappa \approx 3.07$ garantizó estabilidad numérica y minimizó errores de redondeo.

\item Se desarrollaron dos versiones de interpolación: V4 con spline cúbico natural 1D manual (dos etapas) y V4.1 con spline bicúbico de SciPy (simultáneo), demostrando que V4.1 ofrece mayor suavidad y continuidad.

\item Los resultados son físicamente coherentes: aceleración entre obstáculos, recirculación tras vigas, y conservación de masa.

\item El número de Reynolds ($Re = 5.00$) confirmó régimen laminar, validando las hipótesis del modelo.

\item Las líneas de corriente revelaron claramente los patrones de flujo y zonas de recirculación.
\end{enumerate}

Como trabajo futuro se sugiere: estudio de independencia de malla, extensión a Reynolds más altos con modelos de turbulencia, geometrías más complejas, y comparación con métodos de elementos finitos.

\begin{thebibliography}{1}

\bibitem{landau2018}
R. Landau and M. Paez, \emph{Computational Problems for Physics: With Guided Solutions Using Python}. Series in Computational Physics, CRC Press, 2018.

\bibitem{kincaid2002}
D. Kincaid and W. Cheney, \emph{Numerical Analysis: Mathematics of Scientific Computing}. Third Edition, The Sally Series, Pure and Applied Undergraduate Texts, American Mathematical Society, 2002.

\end{thebibliography}

\newpage
\onecolumn

\section*{Anexo A: Derivación de la Ecuación Discretizada}

Partiendo de la ecuación de Navier-Stokes para la componente horizontal:

\begin{equation}
\nu\left(\frac{\partial^2 v_x}{\partial x^2} + \frac{\partial^2 v_x}{\partial y^2}\right) = v_x\frac{\partial v_x}{\partial x} + v_y\frac{\partial v_x}{\partial y} + \frac{1}{\rho}\frac{\partial P}{\partial x}
\end{equation}

Asumiendo presión independiente de densidad y temperatura, y aplicando diferencias finitas centradas con $h=1$:

\begin{multline}
\nu\left[\frac{v^x_{i+1,j} - 2v^x_{i,j} + v^x_{i-1,j}}{h^2} + \frac{v^x_{i,j+1} - 2v^x_{i,j} + v^x_{i,j-1}}{h^2}\right] = \\
v^x_{i,j}\frac{v^x_{i+1,j} - v^x_{i-1,j}}{2h} + v^y_{i,j}\frac{v^x_{i,j+1} - v^x_{i,j-1}}{2h}
\end{multline}

Con $\nu = 1$ y $h = 1$, simplificando y despejando $v^x_{i,j}$:

\begin{multline}
4v^x_{i,j} = v^x_{i+1,j} + v^x_{i-1,j} + v^x_{i,j+1} + v^x_{i,j-1} \\
- 2v^x_{i,j}\left[v^x_{i+1,j} - v^x_{i-1,j}\right] - 2v^y_{i,j}\left[v^x_{i,j+1} - v^x_{i,j-1}\right]
\end{multline}

Finalmente:

\begin{multline}
v^x_{i,j} = \frac{1}{4}\left[v^x_{i+1,j} + v^x_{i-1,j} + v^x_{i,j+1} + v^x_{i,j-1}\right] \\
- \frac{1}{2}v^x_{i,j}\left[v^x_{i+1,j} - v^x_{i-1,j}\right] - \frac{1}{2}v^y_{i,j}\left[v^x_{i,j+1} - v^x_{i,j-1}\right]
\end{multline}

\section*{Anexo B: Cálculo del Jacobiano}

Para calcular la matriz Jacobiana, derivamos la ecuación discretizada respecto a cada variable. Considerando la ecuación para el nodo $(i,j)$:

\begin{equation}
F_{i,j} = v^x_{i,j} - \frac{1}{4}\left[v^x_{i+1,j} + v^x_{i-1,j} + v^x_{i,j+1} + v^x_{i,j-1}\right] + \text{términos no lineales}
\end{equation}

Las derivadas parciales son:

\begin{align}
\frac{\partial F_{i,j}}{\partial v^x_{i,j}} &= 1 + \frac{1}{2}\left[v^x_{i+1,j} - v^x_{i-1,j}\right] + \text{otros términos} \\
\frac{\partial F_{i,j}}{\partial v^x_{i+1,j}} &= -\frac{1}{4} + \frac{1}{2}v^x_{i,j} \\
\frac{\partial F_{i,j}}{\partial v^x_{i-1,j}} &= -\frac{1}{4} - \frac{1}{2}v^x_{i,j} \\
\frac{\partial F_{i,j}}{\partial v^x_{i,j+1}} &= -\frac{1}{4} + \frac{1}{2}v^y_{i,j} \\
\frac{\partial F_{i,j}}{\partial v^x_{i,j-1}} &= -\frac{1}{4} - \frac{1}{2}v^y_{i,j}
\end{align}

Para la implementación computacional, se utiliza la siguiente notación simplificada para los coeficientes de la matriz Jacobiana, correspondientes a la diagonal principal y sus vecinos, tal como se muestra en la siguiente figura:

\begin{figure}[H]
    \centering
    \includegraphics[width=0.8\linewidth]{figuras/notacion_jacobiano.png}
    \caption*{Notación de coeficientes del Jacobiano.}
\end{figure}

Estas cinco derivadas no nulas por ecuación generan la estructura dispersa de banda observada en la Fig. \ref{fig:jacobiano}.

\section*{Anexo C: Ecuaciones Discretizadas en Nodos Específicos}

A continuación se presentan las ecuaciones discretizadas explícitas para 9 nodos representativos de la malla, mostrando la aplicación de las condiciones de frontera (donde $\mathbf{1}$, $\mathbf{0}$ y $\mathbf{V_0}$ representan valores de frontera):

\noindent\textbf{Nodo (1,1)}
\begin{equation*}
v^x_{1,1} = \frac{1}{4} \left\{ v^x_{2,1} + \mathbf{1} + v^x_{1,2} + \mathbf{0} - \frac{1}{2}v^x_{1,1}(v^x_{2,1} - \mathbf{1}) - \frac{1}{2}v^y_{1,1}(v^x_{1,2} - \mathbf{0}) \right\}
\end{equation*}

\noindent\textbf{Nodo (1,20)}
\begin{equation*}
v^x_{1,20} = \frac{1}{4} \left\{ v^x_{2,20} + \mathbf{1} + v^x_{1,21} + v^x_{1,19} - \frac{1}{2}v^x_{1,20}(v^x_{2,20} - \mathbf{1}) - \frac{1}{2}v^y_{1,20}(v^x_{1,21} - v^x_{1,19}) \right\}
\end{equation*}

\noindent\textbf{Nodo (1,39)}
\begin{equation*}
v^x_{1,39} = \frac{1}{4} \left\{ v^x_{2,39} + \mathbf{1} + \mathbf{V_0} + v^x_{1,38} - \frac{1}{2}v^x_{1,39}(v^x_{2,39} - \mathbf{1}) - \frac{1}{2}v^y_{1,39}(\mathbf{V_0} - v^x_{1,38}) \right\}
\end{equation*}

\noindent\textbf{Nodo (174,1)}
\begin{equation*}
v^x_{174,1} = \frac{1}{4} \left\{ v^x_{175,1} + v^x_{173,1} + v^x_{174,2} + \mathbf{0} - \frac{1}{2}v^x_{174,1}(v^x_{175,1} - v^x_{173,1}) - \frac{1}{2}v^y_{174,1}(v^x_{174,2} - \mathbf{0}) \right\}
\end{equation*}

\noindent\textbf{Nodo (200,20)}
\begin{equation*}
v^x_{200,20} = \frac{1}{4} \left\{ v^x_{201,20} + v^x_{199,20} + v^x_{200,21} + v^x_{200,19} - \frac{1}{2}v^x_{200,20}(v^x_{201,20} - v^x_{199,20}) - \frac{1}{2}v^y_{200,20}(v^x_{200,21} - v^x_{200,19}) \right\}
\end{equation*}

\noindent\textbf{Nodo (200,39)}
\begin{equation*}
v^x_{200,39} = \frac{1}{4} \left\{ v^x_{201,39} + v^x_{199,39} + \mathbf{V_0} + v^x_{200,38} - \frac{1}{2}v^x_{200,39}(v^x_{201,39} - v^x_{199,39}) - \frac{1}{2}v^y_{200,39}(\mathbf{V_0} - v^x_{200,38}) \right\}
\end{equation*}

\noindent\textbf{Nodo (399,1)}
\begin{equation*}
v^x_{399,1} = \frac{1}{4} \left\{ \mathbf{0} + v^x_{398,1} + v^x_{399,2} + \mathbf{0} - \frac{1}{2}v^x_{399,1}(\mathbf{0} - v^x_{398,1}) - \frac{1}{2}v^y_{399,1}(v^x_{399,2} - \mathbf{0}) \right\}
\end{equation*}

\noindent\textbf{Nodo (399,20)}
\begin{equation*}
v^x_{399,20} = \frac{1}{4} \left\{ \mathbf{0} + v^x_{398,20} + v^x_{399,21} + v^x_{399,19} - \frac{1}{2}v^x_{399,20}(\mathbf{0} - v^x_{398,20}) - \frac{1}{2}v^y_{399,20}(v^x_{399,21} - v^x_{399,19}) \right\}
\end{equation*}

\noindent\textbf{Nodo (389,39)}
\begin{equation*}
v^x_{389,39} = \frac{1}{4} \left\{ v^x_{390,39} + v^x_{388,39} + \mathbf{V_0} + v^x_{389,38} - \frac{1}{2}v^x_{389,39}(v^x_{390,39} - v^x_{388,39}) - \frac{1}{2}v^y_{389,39}(\mathbf{V_0} - v^x_{389,38}) \right\}
\end{equation*}

\end{document}
